%!TEX root = ../main.tex

\input{parameters/myCommands.tex}
\chapter{Intro - contexte - état de l'art}
    % \include{chapters/soa/introduction/introduction}
    \include{chapters/soa/etat_de_lart/etat_de_lart}
    % \include{chapters/soa/problematique/problematique}
    % \include{chapters/soa/approche/approche}
    % \include{chapters/soa/conclusion/conclusion-francaise}



\chapter{Chaînages dans des domaines fixes}
    \label{section:chapDiaphragme}
     \input{chapters/singleDiaphragm/intro.tex}
    \section{Théorie - description de la méthode utilisée}
        \input{chapters/singleDiaphragm/tout}
    \section{Application : simple diaphragme}
        \input{chapters/singleDiaphragm/aero}
    \section{Comparaison de formulations}
        \input{chapters/singleDiaphragm/comparaisonDesFormulations}
    %%%%%% \section{Étude du mapping}
    %%%%%%     \input{chapters/singleDiaphragm/mapping}
    \section{Conclusions}
        \input{chapters/singleDiaphragm/conclusion}

\chapter{Chaînages avec formulations poreuses}
    \label{section:chapPoreux}
    \input{chapters/dAcd/intro.tex}
    \section{Théorie, formulations faibles}
        % %%%%%%%%%%\subsection{Formulation poreuse}
            \input{chapters/dAcd/theorieIntro}
        \subsection{Formulation poreuse 1}
            \input{chapters/dAcd/poreuxCaro}
        \subsection{Formulation poreuse 2}
            \input{chapters/dAcd/poreuxSourcetronquee}
    \section{Application : profil aérodynamique en conduit}
        \subsection{Description du système}
            \input{chapters/dAcd/dacdDef}
        \subsection{Résultats aérodynamiques}
            \input{chapters/dAcd/aero}
            \clearpage 
        \subsection{Résultats acoustiques}
            \input{chapters/dAcd/acoustique}
    \section{Conclusions}
        \input{chapters/dAcd/conclusion}



\chapter{Résolution acoustique dans un domaine tournant 2D}
    \label{section:chap2D}
    \section{Article }
    \section{Compléments à l'article}
        \subsection{détails d'implémentation}
        Projection, matrice des fonctions de base, matrices élémentaires 
        \subsection{calcul d'erreur tc1 avec beaucoup de nLimit différents}
        \subsection{méga système avec C > 2}
        \subsection{solution analytique 1 qui marche mais pas de ouf }
        \subsection{refaire la démo de Kaltenbacher 2016 pour établir l'équation d'onde ur psi en ALE}
        \subsection{Système global}
            % \input{chapters/2d/systemGlobal}
        \subsection{Solution analytique avec la méthode d'Emmanuel}


\chapter{Résolution acoustique dans un domaine tournant 3D}
    \label{section:chap3D}
    \section{Article }
    \section{Compléments à l'article}
        \subsection{comparaison temporel - fréquentiel cas NZE}
        \subsection{Terme source PCWE}
        \subsection{Brouillon}
            \input{chapters/brouillon}




    
